\documentclass[a4paper,15pt]{article}

\usepackage[T1]{fontenc}
\usepackage[utf8]{inputenc}
\usepackage{graphicx}
\usepackage{xcolor}
\usepackage{mathtools}
\usepackage{esint}

\renewcommand\familydefault{\sfdefault}
\usepackage{tgheros}
\usepackage[defaultmono]{droidmono}

\usepackage{amsmath,amssymb,amsthm,textcomp}
\usepackage{enumerate}
\usepackage{multicol}
\usepackage{tikz}
\usepackage{graphics}
\usepackage{lstlinebgrd}
\usepackage{color}
\usepackage{gensymb}
\usepackage{geometry}
\geometry{total={210mm,297mm},
left=25mm,right=25mm,%
bindingoffset=0mm, top=20mm,bottom=20mm}


\linespread{1.3}

\newcommand{\linia}{\rule{\linewidth}{0.5pt}}

\newtheoremstyle{mytheor}
    {1ex}{1ex}{\normalfont}{0pt}{\scshape}{.}{1ex}
    {{\thmname{#1 }}{\thmnumber{#2}}{\thmnote{ (#3)}}}

\theoremstyle{mytheor}
\newtheorem{defi}{Definition}

% my own titles
\makeatletter
\renewcommand{\maketitle}{
\begin{center}
\vspace{2ex}
{\huge \textsc{\@title}}
\vspace{1ex}
\\
\linia\\
\@author \hfill \@date
\vspace{4ex}
\end{center}
}
\makeatother
%%%

% custom footers and headers
\usepackage{fancyhdr}
\pagestyle{fancy}
\lhead{Page \thepage}
\chead{}
\rhead{Steven Seppala}
\lfoot{Assignment \textnumero{} 1}
\cfoot{Antennas}
\rfoot{ECE 469}
\renewcommand{\headrulewidth}{0pt}
\renewcommand{\footrulewidth}{0pt}
%

% code listing settings
\usepackage{listings}
\lstset{
    language=[ANSI]C,
    basicstyle=\ttfamily\small,
    aboveskip={1.0\baselineskip},
    belowskip={1.0\baselineskip},
    columns=fixed,
    extendedchars=true,
    breaklines=true,
    tabsize=4,
    prebreak=\raisebox{0ex}[0ex][0ex]{\ensuremath{\hookleftarrow}},
    frame=lines,
    showtabs=false,
    showspaces=false,
    showstringspaces=false,
    keywordstyle=\color[rgb]{0.627,0.126,0.941},
    commentstyle=\color[rgb]{0.133,0.545,0.133},
    stringstyle=\color[rgb]{01,0,0},
    numbers=left,
    numberstyle=\small,
    stepnumber=1,
    numbersep=10pt,
    captionpos=t,
    escapeinside={\%*}{*)}
}
	\definecolor{dkgreen}{rgb}{0,0.8,0}
	\definecolor{gray}{rgb}{0.5,0.5,0.5}
	\definecolor{mauve}{rgb}{0.58,0,0.82}

%%%----------%%%----------%%%----------%%%----------%%%
%%%----------Document begins right below this-------%%%
%%%----------%%%----------%%%----------%%%----------%%%
\begin{document}

\title{Assignment \textnumero{} 1}

\author{Steven Seppala}

\date{2 Feb 2015}

\maketitle

\section*{Problem 2.9} 


\[ U(\theta,\phi) = \left\{ 
  \begin{array}{l l}
    1 & \quad \text{0$\degree$ $\leq$ $\theta$ $<$ 20$\degree$  }\\
    .342$\csc(\theta)$ & \quad \text{20$\degree$ $\leq$ $\theta$ $<$ 60$\degree$}\\
    0 & \quad \text {60$\degree$ $\leq$ $\theta$ $\leq$ 180$\degree$}\\
  \end{array} \right.\]\\
  \begin{center}
    $D_o$ = $\frac{U_{max}(\theta,\phi)}{P_{rad}}$\\
    $P_{rad} = \oiint U(\theta,\phi) d\Omega$ \\
    $\implies \int\limits_0^{2\pi} \int\limits_0^\pi U(\theta,\phi) \implies \int\limits_0^{2\pi} 
        [\int\limits_0^{\pi/9} \sin(\theta) d\theta             
        +\int\limits_{\pi/9}^{\pi/3}.342\csc(\theta)\sin(\theta)d\theta  
        + \int\limits_{\pi/3}^{\pi} 0 \sin(\theta)d\theta] \\
        \implies$ 
    $({[-\cos(\theta]\Bigr|_{\substack{\theta=0\\ \theta=\frac{\pi}{9}}}
    + .342 [\theta]\Bigr|_{\substack{\theta=\frac{\pi}{9} \\ \theta=\frac{\pi}{3}}}) * 2\pi$\\
    
    $\implies P_{rad} = 1.879$\\
    $\implies D_o = \frac{4\piU_{max}}{1.879} = \frac{4\pi}{1.879} \implies\\
    D_o = 6.687$ no units \\
    $D_{dBi} = 10 log_{10} D_o \implies \\
    D_{dBi} = 8.253$ dB
    
    \pagebreak
\end{center}
\section*{Problem 2.26}
\begin{enumerate}
\item  C Program source code is in appendix.\\
        Program Output :: \\
        \includegraphics[scale=.7]{integral.png}
\item 
        Krauss Approxamation = $\frac{4\pi}{\Theta_{1r} \Theta_{2r}} \\
        \sqrt\Theta_{1r} = \sqrt\Theta_{2r} :=  U(\theta) = .5 \\
        .5 = [\frac{\sin(\pi\sin(\theta))}{\pi\sin(\theta)}]^2 \implies \theta = .458$\\
        $\frac{4\pi}{.916^2} = 14.9768$ dimensionless \\
        $D_{dB} = 10 log_{10} (14.9768) = 11.75$ dB\\
        
\item
        Tai-Peeira Approxamation = \\
        $\frac{22.181}{2(.916)^2} = 13.217$ dimensionless \\
        $D_{dB} = 10 log_{10} (13.217) = 11.21$ dB\\
\end{enumerate}
    
    \pagebreak
    
\section*{Problem 2.35}
    \begin{center}
    $E^i_w = (\hat{a} x + j \hat{a} y) E_o e^{+jkz}$\\
    $E_a=(\hat{a}+2\hat{a}y)E_1\frac{e^{-jkr}}{r}$\\
    \begin{itemize}
    \item $\Delta_\phi = \phi_x - \phi_y = (2n+1)\frac{\pi}{2}$\\
    Circularly polarized because the y component is 90\degree out of phase with the x component due to its j amplitude. This will make the phase delta of x and y always be odd multiples of $\frac{\pi}{2}$ as well as $E_{x}$ being equal to $E_{y}$.\\
    \item The rotation is clockwise because of the y component has a higher amplitude than the x component and will pull the rotation clockwise.
    \item 
        $E^i_w = \frac{\hat{a}_x+j\hat{a}_y}{\sqrt5} \sqrt5 E_1 \frac{e^{-jkz}}{z} $\\
        $\implies $ 2 components with 0$\degree$ phase difference $\implies$ Linear polarization\\
        
    \item Since it is linear it has no rotation. \\
    
    \item 
        $\hat{\rho}_w = \frac{\hat{a}_x + j\hat{a}_y}{\sqrt2} , \hat{\rho}_a = 
                                    \frac{\hat{a}_x + 2\hat{a}_y}{\sqrt5} $\\
                                    
        PLF = $|\hat{\rho}_w \cdot \hat{\rho}_a|^2 \implies \frac{|1+j|^2}{10} = \frac{5}{10}$\\
        PLF = .5 dimensionless\\
        $PLF_{dB} = 10log_{10}(.5) = -3$ dB\\
    
    \end{itemize}
    \end{center}
    

\section*{Problem 2.40}
    $\hat{\rho_a} = \frac{4\hat{a}_x + j\hat{a}_y}{\sqrt17}$\\
    $\tau = 45\degree$\\
    $PLF = |\frac{4\hat{a}_x + j\hat{a}_y}{\sqrt17} * \frac{\hat{a}_x + j\hat{a}_y}{\sqrt2}|^2$\\
    $ \implies \frac{|4+j|^2}{34} \implies .5$ dimensionless\\
    $10log_{10}(.5) = -3.0103$ dB\\
\pagebreak


\section*{Problem 2.70}
    \begin{enumerate}
    \item Gain = Efficiency * Directivity \\
    $ \varepsilon = \frac{A_{em}}{A_p} \implies $\\
    $ e_o * \varepsilon = \frac{A_{em}}{A_p} * e_o \implies$ \\
    $ 1 = \frac{\frac{\lambda^2}{4\pi} D_o}{10 cm^2}e_o \implies $\\
    $ 10cm^2 = \frac{9cm^2}{4\pi} Gain \implies Gain = \frac{40\pi}{9} $\\
    $ \implies 13.9626$ dimentionless \\
    $ 10log_{10}(13.9626) = 11.45 \implies$ Gain $=11.45$ dB
    
    \item $P_T = A_e * W_i$ \\
    $P_T = 10 \frac{(mW)}{(cm)^2} (10 cm^2 ) (.5) \implies P_T = 50$ mW
    \end{enumerate}
    
    
    
\section*{Problem 2.80}

    \[ U(\theta,\phi) = \left\{ 
    \begin{array}{l l}
    $\cos^4(\theta)$ & \quad \text{0$\degree$ $\leq$ $\theta$ $<$ 90$\degree$  }\\
    0                & \quad \text{90$\degree$ $\leq$ $\theta$ $\leq$ 180$\degree$}\\
    \end{array} \right.\] 
    
    \[\left\{ 
    \begin{array}{l l}
    $1$              & \quad \text{0$\degree$ $\leq$ $\phi$ $\leq$ 360$\degree$  }\\
    \end{array} \right.\] 
    
    
    $A_{em} = \frac{\lambda^2}{4\pi} D_o$ \\
    P_{rad} = $\int\limits_0^{2\pi} \int\limits_0^{2\pi} \cos^4(\theta)\sin(\theta) d\theta d\phi$\\
    $\rightarrow$ calculator $\rightarrow$ \\
    $2\pi [.2] \implies P_{rad} = 1.256$\\
    D_o = $\frac{4\pi}{1.256} = 10.0051$\\
    $\implies \frac{9cm^2}{4\pi} (10.0051) = 7.165 * 10^{-2}$ m^2\\
    
    
\section*{Problem 2.86}

    P_r = e_r * D_r * \frac{\lambda^2}{4\pi} * \frac{P_t D_t}{4\pi R^2}\\
    $\implies P_r = 20db\frac{10w*20dB}{4\pi(50)^2\lambda^2}\frac{\lambda^2}{4\pi}$\\
    $\implies \frac{400dB*10w}{16\pi^22500} = 10$mW \\


\pagebreak

\begin{center}
\section*{APPENDIX} 
    \lstinputlisting{num.c}
    \pagebreak
    *{Makefile for compiling the above code}
    \lstinputlisting{Makefile}

\end{document}
